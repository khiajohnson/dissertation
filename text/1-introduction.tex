%%!TEX root = dissertation.tex

\chapter{Introduction}\label{ch:Introduction}

Bilingual and monolingual linguistic experience differs drastically and is captured in the often-repeated observation that a ``bilingual is NOT the sum of two complete or incomplete monolinguals; rather, [they have] a unique and specific linguistic configuration...a different but complete linguistic entity'' \citep[][p. 6]{grosjean_1989_bilingual}. One of the defining characteristics of a bilingual is a shared phonetic space where both languages are produced and perceived \citep{flege_2021_slmr}. Broadly, this dissertation is concerned with the implications of such a space. What aspects of sound are shared across languages? And, given that speech generally occurs in a communicative context, what is available in the multilingual speech signal to facilitate processes like talker identification and processing in more than one language? The studies presented in this dissertation approach this larger question at different levels in the phonetic space but share motivation in speech perception. This introduction motivates and ties together Chapter 3 (voice quality) and Chapter 4 (sound categories). Each level of phonetic variation has been proposed to account for how listeners can track a talker across languages. 

As such, this introduction will be brief, with a majority of the literature reviewed in the relevant chapters. The introduction proceeds as follows. Section \ref{ch1:sec:bilingualism} gives context to the study of bilingualism in phonetics in broad strokes---that is, who is a bilingual and what are some of the key characteristics that define them. The goal is to set up later chapters, rather than provide a comprehensive discussion. Section \ref{ch1:sec:processing} reviews some of the literature on how multilingual phonetic variation is perceived, emphasizing the Language Familiarity Effect in talker identification and how multilingual listeners process code-switching. Section \ref{ch1:sec:spontaneous} motivates the need to attend to speaking style and argues that spontaneous speech corpora are necessary for the study of multilingual phonetic variation. Lastly, Section \ref{ch1:sec:goals} provides the specific goal or research question for each of the main content chapters---2, 3, and 4.%\ref{ch2:corpus}, \ref{ch3:voice}, and \ref{ch4:uniformity}.

\section{Bilingualism}\label{ch1:sec:bilingualism}

In the most general sense, a bilingual is someone with knowledge of two or more languages \citep{grosjean_1989_bilingual}. Different types of bilinguals are best described on a continuum from first language (L1) to second language (L2) dominance, the bookends of which are monolinguals and replacive bilinguals, with balanced bilinguals in the middle \citep{gertken_2014_blp}. Using a continuum effectively reflects the heterogeneous nature of bilingualism. Dominance and patterns of use are affected by factors such as age of acquisition, immersion environment, frequency, social and communicative context \citep{gertken_2014_blp}. 

Much of the bilingualism literature focuses on early bilinguals in order to draw a distinction with learners. Typically, early bilinguals have learned both languages from early childhood. A common cutoff is age five, or the age at which children begin regularly attending primary school \citep{amengual_2017_type}. Regardless of when bilinguals acquire a language, they do not necessarily use their languages in the same domains. For example, a Cantonese-English bilingual in Vancouver, Canada might use English at school and Cantonese at home. These kinds of division make for markedly different linguistic experience across groups of bilinguals, as well as in comparison to local monolingual populations. Bilingual linguistic experience differs in many other ways, including code-switching \citep{fricke_2016_dimensions}, immersion environment, and formal instruction \citep{fricke_2019_bilingualism}. Each of these factors has a demonstrated effect on speech production. Given the sheer heterogeneity within and across bilingual populations, there may not always be an appropriate monolingual comparison groups. Further, \citet{grosjean_1989_bilingual} and many others have argued that such comparisons are often inappropriate. 

As a result, drawing comparisons between monolinguals and bilinguals may not always be fruitful or even necessarily appropriate, depending on the circumstances. This is reflected by a shift in the literature towards examining bilinguals on a within-population \citep[e.g.,][]{chan_2020_lexically} or within-talker basis \citep[e.g.,][]{simonet_2019_convergence}, or by comparing bilingual populations with different characteristics \citep[e.g.,][]{brown_2009_phonological}. 

One of the major outcomes of this experience, as noted above, is a shared phonetic space, in which bilinguals presumably (i.e., are hypothesized to) use similar voice quality to produce similar sound categories. The literature discussing similarity in voice quality and sound categories will be reviewed in greater detail within Chapters 3 and 4, respectively. % \ref{ch3:voice} and \ref{ch4:uniformity}, respectively. 


%Early bilinguals are a heterogeneous group.
%Fall somewhere in the middle of the spectrum from replacive to %monolingual.
%Cantonese and English, see Chapter 2 for more about the population.

% Same talker, same vocal tract, same mind -- how this leads to the question of shared voice quality.
% A rich literature on crosslinguistic influence in perception and production -- how this motivates chapter 4.
% How much is actually shared?

\section{Processing bilingual talkers}\label{ch1:sec:processing}

Communicating in more than one language doesn't just involve the language produced by bilingual talkers; it also involves how listeners perceive those talkers. While bilingual speech perception is a large and multifaceted field \citep{ingvalson_2014_bilingual}, the clearest motivation comes from the advantage that multilingualism offers in identifying talkers. \citet{orena_2019_identifying} report on a talker identification study with French-English bilingual talkers, in which bilingual listeners---particularly those with language mixing experience---were better able to generalize talker-indexical information learned in English to French and vise versa when compared to monolingual English listeners. \citeauthor{orena_2019_identifying} offer potential explanations for this advantage: ``that there are systematic changes in indexical information...[or] systematic consistencies in linguistic information across bilingual speech'' \citeyearpar[][p. EL308]{orena_2019_identifying}. Bilingual listeners are highly sensitive to subtle differences in acoustic input \citep{ju_2004_falling}. As a result, the presence of systematicity in both talker-indexical and linguistic information---however subtle---would be a boon to bilingual listeners, particularly those with language mixing experience. Such bilinguals would have extensive practice at learning how individual talkers vary as they speak in more than one language and deep familiarity with how a talker varies within and across languages. 

While \citet{orena_2019_identifying} point to some prior work supporting these accounts, convincing evidence remains scarce. This dissertation directly addresses these accounts of bilingual talker identification from the perspective of documenting the speech signal. Chapter 3 examines voice variation---generally considered to reflect talker-indexicality. Chapter 4 focuses on the structure of phonetic category variation---a clear example of linguistic information. While using different methods and addressing different aspects of phonetics, each represents an aspect of the signal that may facilitate crosslinguistic talker identification.

% Other areas to add might include how influence facilitates processing code-switches, how assimilation works in SLM-r, or more on the LFE.

\section{But make it spontaneous}\label{ch1:sec:spontaneous}
% External validity.
% Different findings for different speaking styles.
% Quantity and variety matter in mapping variation.

\section{Thesis goals \& research questions}\label{ch1:sec:goals}
\begin{description}
    \item[Chapter 2] expands on the motivation behind studying spontaneous speech and introduces the SpiCE corpus of spontaneous bilingual speech in Cantonese and English \citep{johnson_2021_spice}. The corpus comprises a substantial portion of this dissertation.
    \item[Chapter 3] focuses on the structure of voice variation. Specifically... X.
    \item[Chapter 4] focuses on the structure of sound categories. Specifically... X.
\end{description}

\endinput % -------------------------------------------------------- %
