%%!TEX root = dissertation.tex

\chapter{Introduction}\label{ch:Introduction}

Bilingual and monolingual linguistic experience differs drastically. This sentiment is captured by the often-repeated observation that a ``bilingual is NOT the sum of two complete or incomplete monolinguals; rather, [they have] a unique and specific linguistic configuration...a different but complete linguistic entity'' \citep[][p. 6]{grosjean_1989_bilingual}. One of the defining characteristics of a bilingual is a shared phonetic space where both languages are produced and perceived \citep{flege_2021_slmr}. Broadly, this dissertation is concerned with the implications of such a space and what aspects of sound are shared across languages. And, given that speech usually occurs in a communicative context, what is available in the multilingual speech signal to facilitate processes like talker identification and processing? The studies presented in this dissertation approach this larger question at different levels in the phonetic space---voice quality (Chapter \ref{ch3:voice}) and sound categories (Chapter \ref{ch4:uniformity}). Yet, they share motivation in speech perception, as various levels of phonetic variation have been proposed to account for how listeners can track a talker across languages. This introduction briefly sets up that shared motivation, tying two otherwise quite different studies together. Given these differences, a majority of the literature is reviewed in the relevant chapters. 

The introduction proceeds as follows. Section \ref{ch1:sec:bilingualism} gives context to the study of bilingualism in phonetics in broad strokes---that is, who is considered to be bilingual and what are some of the key characteristics that define them. The goal is to set up later chapters rather than provide a comprehensive discussion. Section \ref{ch1:sec:processing} reviews some of the literature on how multilingual phonetic variation is perceived, emphasizing talker identification and how multilingual listeners process code-switching. Section \ref{ch1:sec:spontaneous} motivates the need to attend to speaking style and argues that spontaneous speech corpora greatly facilitate the study of multilingual phonetic variation. Lastly, Section \ref{ch1:sec:goals} provides the specific goal or research question for each of the main content chapters---\ref{ch2:corpus}, \ref{ch3:voice}, and \ref{ch4:uniformity}.

\section{Bilingualism}\label{ch1:sec:bilingualism}

The population of interest in this dissertation comprises early bilinguals who are comfortable speaking and comprehending their two main languages---Chapter \ref{ch2:corpus} provides a detailed description. In the most general sense, a bilingual is someone with knowledge of two or more languages \citep{grosjean_1989_bilingual}. This incredibly broad definition includes a diverse range of types of bilinguals. Different kinds of bilinguals are perhaps best described on a continuum from first language (L1) to second language (L2) dominance, the bookends of which are monolinguals and replacive bilinguals, with learners, attriters, and balanced bilinguals in the middle \citep{gertken_2014_blp}. Using a continuum in this way reflects the heterogeneous nature of bilingualism, even if it only captures a particular facet of bilingual competence. Dominance and patterns of use are affected by factors such as age of acquisition, immersion environment, frequency, social and communicative context \citep{gertken_2014_blp}. 

While a spectrum may better reflect the reality of bilingualism, much of the literature focuses on more clearly defined groups such as language learners or early (balanced) bilinguals. Typically, early bilinguals have learned both languages from early childhood. A common cutoff is age five, or the age at which children begin regularly attending primary school \citep{amengual_2017_type}. Regardless of when bilinguals acquire a language, they do not necessarily use their languages to the same extent across different domains. For example, a Cantonese-English bilingual in Vancouver, Canada, might use English at school and Cantonese at home. Bilingual language experience varies is still other ways, including code-switching \citep{fricke_2016_dimensions}, immersion environment \citep{sancier_1997_drift}, and formal instruction \citep{fricke_2019_bilingualism}. Each of these factors has a demonstrated effect on speech production. This variety leads to markedly different linguistic experiences across groups of bilinguals, and as a result, markedly different patterns in speech production. 

Across different kinds of phonetics research in bilingualism, there is a common trend of comparing bilinguals to ``closely matched'' monolingual populations. Given the sheer heterogeneity within and across bilingual populations, there may not always be an appropriate monolingual comparison group. Further, \citet{grosjean_1989_bilingual} and many others have argued that such comparisons are often inappropriate. As a result, drawing comparisons between monolinguals and bilinguals may not always be fruitful or even necessarily appropriate, depending on the circumstances. This is reflected by a shift in the literature towards examining bilinguals on a within-population \citep[e.g.,][]{chan_2020_lexically} or within-talker basis \citep[e.g.,][]{simonet_2019_convergence}, or by comparing bilingual populations with different characteristics \citep[e.g.,][]{brown_2009_phonological}. In all cases, there is a strong push to consider bilinguals as the complete speakers they are.


\section{Processing bilingual talkers}\label{ch1:sec:processing}

Communicating in more than one language doesn't just involve the language produced by bilingual talkers; it also impacts how listeners perceive those talkers. As noted above, one of the major outcomes of bilingualism is a shared phonetic space, in which bilinguals presumably (i.e., are hypothesized to) use similar voice quality to produce similar sound categories.\footnote{The speech production literature discussing similarity in voice quality and sound categories will be reviewed in greater detail within Chapters 3 and 4, respectively.} This shared phonetic space thus also impacts the perception of bilingual talkers, whether the listener is a fellow bilingual or not.

While bilingual speech perception is a large and multifaceted field \citep{ingvalson_2014_bilingual}, the clearest motivation for the studies in this dissertation comes from the advantage that multilingualism offers in identifying talkers. \citet{orena_2019_identifying} report on a talker identification study with French-English bilingual talkers, in which bilingual listeners---particularly those with language mixing experience---were better able to generalize talker-indexical information learned in English to French and vice versa when compared to monolingual English listeners. However, all groups in the study were above chance, suggesting that there are both linguistic and non-linguistic components to talker identification. \citeauthor{orena_2019_identifying} offer several potential explanations for this advantage: ``that there are systematic changes in indexical information...[or] systematic consistencies in linguistic information across bilingual speech'' \citeyearpar[][p. EL308]{orena_2019_identifying}. Bilingual listeners are highly sensitive to subtle differences in acoustic input \citep{ju_2004_falling}. As a result, the presence of systematicity in both talker-indexical and linguistic information---however subtle---would be accessible to bilingual listeners, particularly those with language mixing experience. \citeauthor{orena_2019_identifying} also suggest that the results could be explained because the bilinguals ``were familiar with both languages at the test phase, while the monolinguals were only familiar with one of the languages'' \citeyearpar[][p. EL309]{orena_2019_identifying}, though this account would be difficult to separate from the previous two. 

Regardless of the particular explanation, the bilingual advantage in bilingual talker identification likely arises from their deep familiarity with how talkers vary within and across languages. While these accounts emphasize the linguistic aspect of bilingual competence, there is more to the picture than that. \citet{bullock_2009_sociophonetics} highlight the integral role that sociolinguistic competence plays in accounting for variability in production and that bilinguals have expanded repertoires of forms. That is, a bilingual can produce forms canonical to either language, and they can also diverge, converge, interfere, or hypercorrect depending on the social and cognitive circumstances. This competence is echoed by \citet{kleinschmidt_2018_sociolinguistic} in their integrated account of how social, acoustic, and linguistic variation are perceived. Learning the structure and systematicity of variation, then, is both a linguistic, talker-indexical, and social venture. This dissertation investigates the role of the former two but acknowledges the importance of attending to and considering the social component as well.

While \citet{orena_2019_identifying} point to some prior work supporting the talker-indexical and linguistic accounts of bilingual talker identification, convincing evidence remains scarce. This dissertation directly addresses these accounts from the perspective of documenting the speech signal. Chapter 3 examines voice variation---generally considered to reflect talker-indexicality. Chapter 4 focuses on the structure of phonetic category variation---a clear example of linguistic information. While using different methods and addressing different aspects of phonetics, each represents an aspect of the signal that may facilitate crosslinguistic talker identification. 

\section{Variability in conversational speech}\label{ch1:sec:spontaneous}

One of the primary goals of this dissertation is to document and investigate the structure of phonetic variation. While variability is inherent to the speech signal \citep[cf. the lack of invariance problem][]{liberman_1967_perception}, spontaneous speech encompasses a greater degree of variability than other speaking styles \citep[e.g., reduction phenomena:][]{johnson_2004_massive}. Additionally, as the motivation for the studies in Chapters \ref{ch3:voice} and \ref{ch4:uniformity} stems from listeners' ability to identify talkers in more than one language, using conversational speech also supports the external validity of this dissertation. Conversational speech better reflects the range of forms that people use and perceive in their daily lives \hl{cite something}. Additionally, given the range of variability, it is also necessary to study a large enough sample such that it comprises the range of variation for the particular communicative situation. In a similar vein, \citet{tanner_2020_english} argue that large-scale corpus studies are uniquely valuable for understanding phonetic variation. For this reason, this dissertation leverages the study of conversational speech data from a sufficiently large speech corpus. The corpus, along with further motivation for its use, is given in Chapter \ref{ch2:corpus}.

\section{Thesis goals \& research questions}\label{ch1:sec:goals}

While each of the main content chapters in this dissertation is united by common motivation, each has a unique focus. These are as follows:

\begin{description}
    \item[Chapter 2] expands on the motivation behind studying spontaneous speech and introduces the SpiCE corpus of spontaneous bilingual speech in Cantonese and English \citep{johnson_2021_spice}. The corpus comprises a substantial portion of this dissertation.
    \item[Chapter 3] focuses on the structure of voice variation. In broad terms, it asks: Do bilinguals have the same voice in each of their languages? More specifically, do bilinguals exhibit similar spectral properties and lower-dimensional structure in their voice across each language they speak? Chapter 3 also addresses a methodological question regarding the sample size necessary for the methods used. 
    \item[Chapter 4] focuses on the structure of sound categories. In broad terms, it asks: Do bilinguals produce long-lag stops in the same way in each of their languages? More specifically, it describes the structure and sources of variation in how bilinguals produce voice-onset time in conversational speech.
\end{description}

\endinput % -------------------------------------------------------- %
