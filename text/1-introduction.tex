%%!TEX root = dissertation.tex

\chapter{Introduction}
\label{ch:Introduction}

Bilingual and monolingual linguistic experience differs drastically and is captured in the often-repeated observation that ``a bilingual is not simply two monolinguals in one'' (Grosjean, 1989). A defining characteristic of bilingual speech is a shared phonetic space in which both languages are produced and perceived \citep{}. Broadly, this dissertation is concerned with the consequences of this shared phonetic space. What aspects of sound are shared across languages? What is available in the speech signal to facilitate processes like multilingual talker identification and discrimination? While the studies presented in this dissertation approach this big picture question at different levels---voice quality and sound categories---they share motivation in speech perception. This introduction seeks to motivate and tie together the studies described in Chapters 3 and 4. 

\section{Bilingualism}
Early bilinguals are a heterogeneous group.
Fall somewhere in the middle of the spectrum from replacive to monolingual.
Cantonese and English, see Chapter 2 for more about the population.

\section{Shared Phonetic Space} % the premise
Same talker, same vocal tract, same mind -- how this leads to the question of shared voice quality.
A rich literature on crosslinguistic influence in perception and production -- how this motivates chapter 4.
How much is actually shared?

\section{Perceiving multilingualism}
What we know about perceiving speech from the same individual in more than one language.
LFE in bilingualism.
SLM-r suggests we can expect some degree of assimilation, though chapter 4 problematizes this assumption.
Influence facilitates processing a code-switch.

\section{But make it spontaneous}
External validity.
Different findings for different speaking styles.
Quantity matters in mapping variation.

\section{Thesis goals \& research questions}
Chapter 2 introduces the corpus developed as a part of this dissertation.
Chapter 3 asks what is shared at the level of voice quality. Specifically...X.
Chapter 4 asks what is shared at the level of sound categories. Specifically...X.


\endinput % -------------------------------------------------------- %
