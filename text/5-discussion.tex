%%!TEX root = dissertation.tex
\chapter{Discussion \& Conclusion}\label{ch5:discussion}

What are the consequences of a shared phonetic space in the linguistic systems of bilinguals? What is shared? What is kept separate? And, how can methods couched in the study of crosslinguistic influence provide insight into these areas? These questions sit at the core of this dissertation and are approached from two different angles in Chapters \ref{ch3:voice} and \ref{ch4:uniformity}, using the data set described in Chapter \ref{ch2:corpus}. While this dissertation focuses on describing and understanding the speech signal in production, the uniting motivation comes from how the signal is perceived. As such, this chapter proceeds as follows. Section \ref{ch5:sec:recap} recapitulates the main points of the content chapters of this thesis, emphasizing the conclusions that are unique to the chapter. Section \ref{ch5:sec:discussion} dives into a more general discussion, highlighting how the studies conspire together to inform a broader understanding of how variation is structured in bilingual speech production. Additionally, implications for perception are considered. Section \ref{ch5:sec:limitations} makes note of limitations and caveats, and Section \ref{ch5:sec:directions} flags current perception research aimed at complimenting this dissertation. Lastly, Section \ref{ch5:sec:conclusion} concludes by summarizing the key contributions of this dissertation to the fields of phonetics, psycholinguistics, and bilingualism.

\section{Recap}\label{ch5:sec:recap}

Chapter \ref{ch2:corpus} introduces a new speech corpus, developed as a part of this dissertation. The SpiCE corpus of Speech in Cantonese and English comprises high-quality recordings and transcripts of sentence reading, storyboard narration, and conversational interviews in each language. All participants were early bilinguals and members of the heterogeneous bilingual speech community in Vancouver, BC, Canada. Chapter \ref{ch2:corpus} documents the motivation, design, and procedures used in the creation of SpiCE. Additionally, a detailed description of the participants is provided. SpiCE is an open-access corpus freely available to anyone interested in the data---researchers, developers, hobbyists, and the general public \citep{johnson_2021_spice}. On its own, SpiCE represents a major contribution to the study of bilingual speech production. 

Chapter \ref{ch3:voice} describes a study on the structure of acoustic voice variation within and across languages for the talkers in the SpiCE corpus. Using a wide array of source and filter acoustic measurements on voiced speech in the conversational interviews, Chapter \ref{ch3:voice} addresses crosslinguistic similarity in three ways. First, the distributions of each measurement were compared across languages on a by-talker basis using Cohen's \textit{d}. The vast majority of comparisons resulted in trivial differences, indicating that talkers were mostly internally consistent. Where consistent differences emerged, they mostly aligned with prior work---Cantonese tended to have lower fundamental frequency and be associated with breathier (or less creaky) voice quality than English \citep{ng_2012_ltas}. 

Second, a series of principal components analyses (PCAs) were run for each talker and language pair. In broad terms, the PCAs bore remarkable similarities in component structure and variance accounted for, regardless of talker and language, given prior work in this domain \citep{lee_2019_acoustic, lee_2019_spontaneous, lee_2020_language}. The PCAs were then subjected to canonical correlation analyses to elucidate how much of the lower dimensional structure in one PCA could be accounted for by the other PCA---that is, how much redundancy there is between two PCAs---and vice versa. The result of this analysis clearly demonstrates that talkers bear the most similarity to themself across languages, compared to any across-talker comparison. While there is some variation in the degree of similarity, the takeaway from this chapter is that voices can largely be thought of as ``auditory faces.''

Chapter \ref{ch4:uniformity} presents a second corpus study, focused on describing and analyzing the structure of phonetic category variation within and across languages for long-lag stops. Leveraging the uniformity framework \citep{chodroff_2017_structure}, Chapter \ref{ch4:uniformity} demonstrates that there is some structure to the relationship between voice-onset time patterns, but that the account is far less compelling when compared to prior work on English \citep{chodroff_2017_structure, chodroff_2019_l2}. Talkers were wildly inconsistent with respect to the expected ordinal relationships between means for the stop categories within languages. That is, very few talkers produced /p/ with shorter voice onset time (VOT) than /t/ or /k/, and likewise between /t/ and /k/. The second phase of the analysis considered pairwise correlations of category means for VOT within and across languages. Again, the results were far from compelling. While there were consistent moderate correlations for within-language comparisons---especially for English---the across-language correlations were weaker or non-significant. Their presence indicates some degree of structure but does not make for tidy conclusions. 

Chapter \ref{ch4:uniformity} ends with a Bayesian linear mixed-effects model with two primary goals. First, estimate the effect of language while accounting factors known to influence VOT. Second, evaluate the sources of variability within the model. The model demonstrates a small but consistent effect, such that English stops are longer than their Cantonese counterparts. The model also indicates that differences between talkers and words account for far more variation than differences in language and place of articulation effects. Along with the ordinal relationships and correlation analysis, Chapter \ref{ch4:uniformity} depicts both talker and language influences on the structure of VOT. 

\section{General discussion}\label{ch5:sec:discussion}

Each of the chapters includes a discussion that deals with topics central to the study at hand. This general discussion focuses on the bigger picture and implications that the two studies have for perception, as outlined in Chapter \ref{ch1:intro}. While there is substantial similarity across languages in both voice variability and the structure of long-lag stops, each study leaves room for both language and talker-indexical influences. 

In Chapter \ref{ch3:voice}, the dual influences of talker and language show up in the redundancy analysis. While talkers exhibit the greatest degree of similarity when compared to themselves across languages,  no talker exhibits perfect redundancy. The variability in this metric suggests influence from non-talker-indexical sources, which could be linguistic or reflect social dynamics and positioning. Yet while Chapter \ref{ch3:voice} does not rule out any particular type of influence, there seems to be a much clearer role for talker-indexical features, given the high degree of within-talker similarity across languages. This observation is further reflected in the comparisons of each acoustic measurement via Cohen's \textit{d}. Just over a third of the talkers in the SpiCE corpus maintain a crosslinguistic difference---in the same direction---for measures associated with pitch and non-modal voice quality. Some prior work has attempted to account for differences in fundamental frequency via the presence or absence of lexical tone. These studies, however, are not consistent with one another---some suggest lexical tone leads to lower F0 \citep{ng_2012_ltas}, and others to higher F0 \citep{lee_2017_bilingual, keating_2012_f0}. It may be the case that a particular tone system impacts a talker's F0 profile, but the evidence is not yet compelling for this argument. 

If the differences of the present study were due to linguistic differences, then it would be surprising that only a third of talkers exhibited the difference. A more likely account is one that invokes social factors and how individuals express their identity in each language \citep{loveday_1981_pitch, voigt_2016_between}. The lack of a clear role for linguistic influences here is further compounded by the behavior of the acoustic measurements typically associated with linguistic attributes. For example, while the patterning of F2 differences across languages largely reflects expectations around the distributions of back vowels in English and Cantonese, the direction of the effect is not consistent across talkers. While the expected difference is present for some, the uncontrolled nature of spontaneous speech makes it challenging to draw a conclusion that reflects the group as a whole.

The dual influences of talker and language are more apparent in Chapter \ref{ch4:uniformity}. This outcome is perhaps not surprising, given how Chapter \ref{ch3:voice} examined voice quality and Chapter \ref{ch4:uniformity} homes in on speech sound categories. The former can be used in both linguistic and non-linguistic manners, while the latter is decidedly a linguistic level of structure. Influence from both talker and language shows up in the juxtaposition between moderate correlations and the difference in VOT across languages. The moderate correlations for homorganic cross-language pairs indicate some level of uniformity in production. Yet, at the same time, the crosslinguistic difference whereby English is characterized by longer VOT reflects a language-based influence. 

Chapter \ref{ch1:intro} framed this dissertation as being concerned with the consequences of a shared phonetic space, particularly with regard to how the speech signal facilitates processing and identifying multilingual talkers. If talkers are to be consistently identified before and after a language switch, then the speech signals in each language must resemble one another in some way. Chapters \ref{ch3:voice} and \ref{ch4:uniformity} grapple with how that resemblance plays out in production. Recall the summary of \citet{orena_2019_identifying} in Chapter \ref{ch1:intro}, in which bilingual listeners outperformed monolingual listeners in an identification study featuring bilingual talkers. The authors presented several possible accounts for why bilinguals had an advantage in the task, even if performance was above chance across the board. Two of the accounts appeal to systematicity across languages---\citet{orena_2019_identifying} suggest that bilingual listeners are sensitive to systematic \textit{changes} in talker-indexical information and systematic \textit{consistencies} in the linguistic signal. 

How, then, do these accounts fare in the context of this dissertation? Put differently, is the assumed (or proposed) systematicity present in the bilingual speech signal? Chapter \ref{ch3:voice} presents a strong case for the structure of acoustic voice variation and a wide variety of specific acoustic measurements. Even in cases where a subset of participants show a non-trivial difference across languages, few if any differences are large, especially when compared to across-talker differences. In essence, Chapter \ref{ch3:voice} supports the talker-indexical account, albeit one where there are more likely to be systematic similarities rather than changes. It does not, however, discount a linguistic account. 

Chapter \ref{ch4:uniformity} offers support for both accounts, though again, it seems to be one of consistency in talker-indexical systematicity and systematic differences in the linguistic system. Note, however, that while the small across-language difference is meaningful regarding what it tells us about the production system, it is not necessarily meaningful to listeners. Such a small difference thus further supports the talker-indexical view of what listeners use to identify talkers across languages. \hl{this paragraph needs a lot of work}

Returning to the broad question of what language share in phonetic space, this dissertation provides a peek behind the curtain. Languages appear to share a lot in voice quality, despite distributional differences in the segment inventories and different roles for suprasegmental linguistic components. While such differences may be apparent in shorter stretches of speech---as suggested by the passage length analysis in Section \ref{ch3:sec:passagelength}---over time, talkers appear to cover their full range. This coverage indicates that different languages make similar use of an individual's full range of acoustic voice variation and exhibit similar patterns of variation in the long run. 

Languages also share phonetic category structure, albeit to a lesser extent. Chapter \ref{ch4:uniformity} demonstrates that talkers with longer VOT in one language tend to have longer VOT in the other language, even though a small distinction between the two languages is maintained. This outcome echoes results for speech rate, in which late bilinguals who are fast in their first language also tend to be fast talkers in their second language. While this relationship between languages is not as simple as saying individuals use the same thing in each language, it does demonstrate a certain degree of shared-ness while simultaneously highlighting the complexity of factors conspiring together to produce the acoustic signal. 

What is clear from both studies is that there is more than enough shared structure for listeners to use in identifying bilingual talkers. The bilingual advantage in this domain could stem from the less-than-perfect within-talker across-language similarity in voice variation and bilinguals' familiarity with how voices might deviate due to social and linguistic reasons. Similarly, it could stem from bilinguals' familiarity with the variety of forms in which a particular category can be produced. While there are clear examples of this kind of sociolinguistically informed variation for initial stops in other language pairs \citep{bullock_2009_sociophonetics}, there is also evidence of this kind of metalinguistic knowledge for different categories in Cantonese-English bilinguals. A recent example juxtaposes the production of word-final stops by the talkers in the SpiCE corpus \citep{johnson_2021_language} and a lab-based study of a similar population \citep{polinsky_2018_heritage}. The corpus study demonstrates variability that skews towards Cantonese-like unreleased stops. The lab-based study, conversely, gives evidence for hypercorrection towards longer releases. By adopting the perspective of \citet{bullock_2009_sociophonetics}, this discrepancy is readily explainable via metalinguistic awareness and how bilinguals use their language in different ways when talking to their peers versus speaking in formal, monolingual, lab-based settings. The point of bringing this example up is to illustrate that bilingual listeners are not only sensitive to the fine-grained acoustics \citep{ju_2004_falling}, they are also sensitive to the ways that form varies according to the communicative context. In sum, there are both systematic similarities and differences available in the signals for listeners---and bilingual listeners in particular---to use in identification.

\section{Limitations}\label{ch5:sec:limitations}

As with any study, the results presented in this dissertation are necessarily tempered by some limitations and leave some amount of variation unaccounted for. The simplest form of limitation arises from methodological decisions and is a point touched on in Chapters \ref{ch3:voice} and \ref{ch4:uniformity}. Both studies use corpus methods with exclusionary criteria and minimal manual inspection. In Chapter \ref{ch3:voice} this takes the form of using an automated approach to identify voiced portions of speech and a set of exclusionary criteria to discard likely errors. In Chapter \ref{ch4:uniformity} this takes the form of relying on forced alignment, refinement via automated methods, and exclusionary criteria. Such approaches allow for the studies to be done at a large scale but also mean that some degree of error is inevitable. The samples may have include items that do not reflect the target. For example, some erroneous VOT measurements may have evaded the exclusionary criteria in Chapter \ref{ch4:uniformity}---without a rigorous manual check or manual transcription of the SpiCE corpus, the true extent of this problem will remain unknown. While it is outside the scope of this dissertation to perform such a check, prior corpus work with less stringent exclusionary criteria indicates that the error rate is relatively small \citep{chodroff_2017_structure}.

The population studied here also presents a limitation on the extent to which the results of this dissertation can be generalized to other bilingual populations. As summarized in Section \ref{ch1:sec:bilingualism}, there is enormous variation between and among bilingual populations---the talkers in the SpiCE corpus are no exception. As described in detail in Chapter \ref{ch2:corpus}, the population studied in this dissertation represents a heterogeneous group of early Cantonese-English bilinguals. While some factors were carefully controlled for in participant recruitment, others were not. On the one hand, participants were 18-35 years old at recording, comfortable conversing in Cantonese and English, and began learning both languages before age five. On the other hand, most participants had some knowledge of at least one additional language (e.g., Mandarin, French, etc.) and varied in their family's current or historical roots in the Cantonese-speaking homeland.  

While variability is an inherent part of the Cantonese-speaking community in Vancouver, BC, Canada---and thus justifiably included in the corpus---it does make comparisons with other bilingual communities somewhat challenging. Such comparisons are also rendered more difficult by the rather unique position of the speech community. Cantonese is reported as the ``mother tongue'' of some eight percent of Metro Vancouver census respondents \citep{statistics_2017_proportion}. Given the overall population, Cantonese is an incredibly visible minority language in the region. As argued in \citet{chan_2020_lexically}, this population likely has more access to Cantonese than bilingual communities in other English-dominant societies \citep[e.g.,][]{bruggeman_2019_l1}. While there is much more to say on this topic, a detailed study is left for future work. Further, while this corpus could be used to explore sources of variation based on different aspects of the participants' demographics, a lack of control on many of the potentially relevant parameters renders such approaches speculative. 

\section{Current and future directions}\label{ch5:sec:directions}

There is another kind of ``limitation'' baked into this dissertation's partially exploratory design. While the focus remains on describing and accounting for variation in speech production, the motivation arises from perception. This disconnect means that the questions relating to speech perception are not answered. Rather, the results presented in this dissertation generate hypotheses for how bilingual talkers are perceived and identified. This section aims to outline what those hypotheses are and flag current research addressing them. 

% Given the variability of 

% Ch 3 Hypotheses: 
% Within-talker across-language redundancy will facilitate talker identification across languages.
% Global shifts in correlates of pitch and spectral tilt are prevalent if not ubiquitous -- test whether or not listeners can adapt to these shifts better than variables that don't have a consistent-when-present direction.

% Ch 4 Hypotheses: 
% Uniformity decreases as speech style becomes progressively more casual.
% Given high between-talker variability, all listeners will benefit from congruent VOT, but bilingual listeners will be better at learning systematic difference -- piggybacks on code-switching work.

% Current work in the Speech in Context Laboratory is looking at
%  talker identification with spontaneous speech
% four female talkers
% many types of bilingual listeners
% preliminary results suggest that knowledge of a tone language (Cantonese, Mandarin, or Punjabi) leads to a better overall ability to generalize from English to Cantonese, which tracks with tone language speakers having better pitch perception \citep{xie_2015_impact}
% while early, this supports a view in which listeners access and learn systematicity in the signal and where better pitch and pitch variability perception give a leg up
% talker discrimination with spontaneous speech
% A perceptual analog to redundancy---that is, does canonical redundancy align with confusability?
% currently running 


\section{Conclusion}\label{ch5:sec:conclusion}

Speech is variable, and learning a new talker can be characterized as learning how that talker varies. This dissertation focuses on comparing systematicity across languages to understand how such structure might facilitate processes like multilingual talker identification. The results presented here demonstrate the presence of systematicity at two levels---acoustic voice variation and how long-lag stop series manifest. This structure shows evidence of both talker-indexical and linguistic influences and generates a multitude of hypotheses for future work. 

There is a balance between variation and structure at every level---talker, language, linguistic units, voice quality, and more. Working with spontaneous speech corpora is one of the better ways to gain an appreciation for this observation. It would not be possible to do this kind of research without data, and one of the largest and lasting contributions of this dissertation is the SpiCE corpus. While this dissertation just scratches the surface of what can be done with SpiCE, making the data available will help push our understanding of bilingual speech production forward. 

\endinput % -------------------------------------------------------- %
