%%!TEX root = dissertation.tex
\chapter{Discussion \& Conclusion}\label{ch5:discussion}

What are the consequences of a shared phonetic space in the linguistic systems of bilinguals? What is shared? What is kept separate? And, how can methods couched in the study of crosslinguistic influence provide insight into these areas? These questions sit at the core of this dissertation and are approached from two different angles in Chapters \ref{ch3:voice} and \ref{ch4:uniformity}, using the data set described in Chapter \ref{ch2:corpus}. While this dissertation focuses on describing and understanding the speech signal in production, the uniting motivation comes from how the signal is perceived. As such, this chapter proceeds as follows. Section \ref{ch5:sec:recap} recapitulates the main points of the content chapters of this thesis, emphasizing the conclusions that are unique to the chapter. Section \ref{ch5:sec:discussion} dives into a more general discussion, highlighting how the studies conspire together to inform a broader understanding of how variation is structured in bilingual speech production. Additionally, implications for perception are considered. Section \ref{ch5:sec:limitations} makes note of limitations and caveats, and Section \ref{ch5:sec:directions} flags current perception research aimed at complimenting this dissertation. Lastly, Section \ref{ch5:sec:conclusion} concludes by summarizing the key contributions of this dissertation to the fields of phonetics, psycholinguistics, and bilingualism.

\section{Recap}\label{ch5:sec:recap}

Chapter \ref{ch2:corpus} introduces a new speech corpus, developed as a part of this dissertation. The SpiCE corpus of Speech in Cantonese and English comprises high-quality recordings and transcripts of sentence reading, storyboard narration, and conversational interviews in each language. All participants were early bilinguals and members of the heterogeneous bilingual speech community in Vancouver, BC, Canada. Chapter \ref{ch2:corpus} documents the motivation, design, and procedures used in the creation of SpiCE. Additionally, a detailed description of the participants is provided. SpiCE is an open-access corpus freely available to anyone interested in the data---researchers, developers, hobbyists, and the general public \citep{johnson_2021_spice}. On its own, SpiCE represents a major contribution to the study of bilingual speech production. 

Chapter \ref{ch3:voice} describes a study on the structure of acoustic voice variation within and across languages for the talkers in the SpiCE corpus. Using a wide array of source and filter acoustic measurements on voiced speech in the conversational interviews, Chapter \ref{ch3:voice} addresses crosslinguistic similarity in three ways. First, the distributions of each measurement were compared across languages on a by-talker basis using Cohen's \textit{d}. The vast majority of comparisons resulted in trivial differences, indicating that talkers were mostly internally consistent. Where consistent differences emerged, they mostly aligned with prior work---Cantonese tended to have lower fundamental frequency and be associated with breathier (or less creaky) voice quality than English \citep{ng_2012_ltas}. 

Second, a series of principal components analyses (PCAs) were run for each talker and language pair. In broad terms, the PCAs bore remarkable similarities in component structure and variance accounted for, regardless of talker and language, given prior work in this domain \citep{lee_2019_acoustic, lee_2019_spontaneous, lee_2020_language}. The PCAs were then subjected to canonical correlation analyses to elucidate how much of the lower dimensional structure in one PCA could be accounted for by the other PCA---that is, how much redundancy there is between two PCAs---and vice versa. The result of this analysis clearly demonstrates that talkers bear the most similarity to themself across languages, compared to any across-talker comparison. While there is some variation in the degree of similarity, the takeaway from this chapter is that voices can largely be thought of as ``auditory faces.''

Chapter \ref{ch4:uniformity} presents a second corpus study, focused on describing and analyzing the structure of phonetic category variation within and across languages for long-lag stops. Leveraging the uniformity framework \citep{chodroff_2017_structure}, Chapter \ref{ch4:uniformity} demonstrates that there is some structure to the relationship between voice-onset time patterns, but that the account is far less compelling when compared to prior work on English \citep{chodroff_2017_structure, chodroff_2019_l2}. Talkers were wildly inconsistent with respect to the expected ordinal relationships between means for the stop categories within languages. That is, very few talkers produced /p/ with shorter voice onset time (VOT) than /t/ or /k/, and likewise between /t/ and /k/. The second phase of the analysis considered pairwise correlations of category means for VOT within and across languages. Again, the results were far from compelling. While there were consistent moderate correlations for within-language comparisons---especially for English---the across-language correlations were weaker or non-significant. Their presence indicates some degree of structure but does not make for tidy conclusions. 

Chapter \ref{ch4:uniformity} ends with a Bayesian linear mixed-effects model with two primary goals. First, estimate the effect of language while accounting factors known to influence VOT. Second, evaluate the sources of variability within the model. The model demonstrates a small but consistent effect, such that English stops are longer than their Cantonese counterparts. The model also indicates that differences between talkers and words account for far more variation than differences in language and place of articulation effects. Along with the ordinal relationships and correlation analysis, Chapter \ref{ch4:uniformity} depicts both talker and language influences on the structure of VOT. 

\section{General discussion}\label{ch5:sec:discussion}

Each of the chapters includes a discussion that deals with topics central to the study at hand. This general discussion focuses on the bigger picture and implications that the two studies have for perception, as outlined in Chapter \ref{ch1:intro}. While there is substantial similarity across languages in both voice variability and the structure of long-lag stops, each study leaves room for both language and talker-indexical influences. 

In Chapter \ref{ch3:voice}, the dual influences of talker and language show up in the redundancy analysis. While talkers exhibit the greatest degree of similarity when compared to themselves across languages,  no talker exhibits perfect redundancy. The variability in this metric suggests influence from non-talker-indexical sources, which could be linguistic or reflect social dynamics and positioning. Yet while Chapter \ref{ch3:voice} does not rule out any particular type of influence, there seems to be a much clearer role for talker-indexical features, given the high degree of within-talker similarity across languages. This observation is further reflected in the comparisons of each acoustic measurement via Cohen's \textit{d}. Just over a third of the talkers in the SpiCE corpus maintain a crosslinguistic difference---in the same direction---for measures associated with pitch and non-modal voice quality. Some prior work has attempted to account for differences in fundamental frequency via the presence or absence of lexical tone. These studies, however, are not consistent with one another---some suggest lexical tone leads to lower F0 \citep{ng_2012_ltas}, and others to higher F0 \citep{lee_2017_bilingual, keating_2012_f0}. It may be the case that a particular tone system impacts a talker's F0 profile, but the evidence is not yet compelling for this argument. 

If the differences of the present study were due to linguistic differences, then it would be surprising that only a third of talkers exhibited the difference. A more likely account is one that invokes social factors and how individuals express their identity in each language \citep{loveday_1981_pitch, voigt_2016_between}. The lack of a clear role for linguistic influences here is further compounded by the behavior of the acoustic measurements typically associated with linguistic attributes. For example, while the patterning of F2 differences across languages largely reflects expectations around the distributions of back vowels in English and Cantonese, the direction of the effect is not consistent across talkers. While the expected difference is present for some, the uncontrolled nature of spontaneous speech makes it challenging to draw a conclusion that reflects the group as a whole.

The dual influences of talker and language are more apparent in Chapter \ref{ch4:uniformity}. This outcome is perhaps not surprising, given how Chapter \ref{ch3:voice} examined voice quality and Chapter \ref{ch4:uniformity} homes in on speech sound categories. The former can be used in both linguistic and non-linguistic manners, while the latter is decidedly a linguistic level of structure. Influence from both talker and language shows up in the juxtaposition between moderate correlations and the difference in VOT across languages. The moderate correlations for homorganic cross-language pairs indicate some level of uniformity in production. Yet, at the same time, the crosslinguistic difference whereby English is characterized by longer VOT reflects a language-based influence. 

Chapter \ref{ch1:intro} framed this dissertation as being concerned with the consequences of a shared phonetic space, particularly with regard to how the speech signal facilitates processing and identifying multilingual talkers. If talkers are to be consistently identified before and after a language switch, then the speech signals in each language must resemble one another in some way. Chapters \ref{ch3:voice} and \ref{ch4:uniformity} grapple with how that resemblance plays out in production. Recall the summary of \citet{orena_2019_identifying} in Chapter \ref{ch1:intro}, in which bilingual listeners outperformed monolingual listeners in an identification study featuring bilingual talkers. The authors presented several possible accounts for why bilinguals had an advantage in the task, even if performance was above chance across the board. Two of the accounts appeal to systematicity across languages---\citet{orena_2019_identifying} suggest that bilingual listeners are sensitive to systematic \textit{changes} in talker-indexical information and systematic \textit{consistencies} in the linguistic signal. 

How, then, do these accounts fare in the context of this dissertation? Put differently, is the assumed (or proposed) systematicity present in the bilingual speech signal? Chapter \ref{ch3:voice} presents a strong case for the structure of acoustic voice variation and a wide variety of specific acoustic measurements. Even in cases where a subset of participants show a non-trivial difference across languages, few if any differences are large, especially when compared to across-talker differences. In essence, Chapter \ref{ch3:voice} supports the talker-indexical account, albeit one where there are more likely to be systematic similarities rather than changes. It does not, however, discount a linguistic account. 

Chapter \ref{ch4:uniformity} offers support for both accounts, though again, it seems to be one of consistency in talker-indexical systematicity and systematic differences in the linguistic system. Note, however, that while the small across-language difference is meaningful regarding what it tells us about the production system, it is not necessarily meaningful to listeners. Such a small difference thus further supports the talker-indexical view of what listeners use to identify talkers across languages. \hl{this paragraph needs a lot of work}

% Conclusions, or what Chapters 3 and 4 tell us about:
% how variation is structured across languages
% what languages share
% where languages differ
% what exists in the signal for perception to rely on
% answers to Orena et al.'s speculations

\section{Limitations}\label{ch5:sec:limitations}

% Limitations and caveats
% Based on a heterogeneous group of 34 bilinguals, controlled tightly in some ways but not in others
% Variable behavior not accounted for -- lots of individual differences
% Forced alignment
% Methods more exploratory
% Generates hypotheses and predictions for perception but does not necessarily answer them

\section{Current and future directions}\label{ch5:sec:directions}
% Current perception research
% Promising directions

\section{Conclusion}\label{ch5:sec:conclusion}Take-home points:
% Learning a voice is to learn how it varies.
% The same goes for speech sound categories, suggesting Bullock & Toribio were really onto something.
% Identification likely relies heavily on voice variation structure, but systematic shifts (i.e., in F0 or speech categories) are not off the table.
% just scratching the surface here, but data sets like SpiCE will help push the field forward


\endinput % -------------------------------------------------------- %
