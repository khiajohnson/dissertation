%%!TEX root = dissertation.tex

\chapter{Discussion \& Conclusion}
\label{ch:Discussion}

What are the consequences of a shared phonetic space in the linguistic systems of bilinguals? What is shared? What is kept separate? And, how can methods couched in the study of crosslinguistic influence provide insight into these areas? These questions sit at the core of this dissertation and are approached from two different angles in Chapters \ref{ch3:voice} and \ref{ch4:uniformity}, using the data set described in Chapter \ref{ch2:corpus}. While this dissertation focuses on describing and understanding the speech signal in production, the uniting motivation comes from how the signal is perceived. As such, this chapter proceeds as follows. Section \ref{ch5:sec:recap} recapitulates the main points of the content chapters of this thesis, emphasizing the conclusions that are unique to the chapter. Section \ref{ch5:sec:discussion} dives into a more general discussion, highlighting how the studies conspire together to inform a broader understanding of how variation is structured in bilingual speech production. Additionally, implications for perception are considered. Section \ref{ch5:sec:limitations} makes note of limitations and caveats, and Section \ref{ch5:sec:directions} flags current perception research aimed at complimenting this dissertation. Lastly, Section \ref{ch5:sec:conclusion} concludes by summarizing the key contributions of this dissertation to the fields of phonetics, psycholinguistics, and bilingualism.

\section{Recap}\label{ch5:sec:recap}

Chapter \ref{ch2:corpus} introduces a new speech corpus, developed as a part of this dissertation. The SpiCE corpus of Speech in Cantonese and English comprises high-quality recordings and transcripts of sentence reading, storyboard narration, and conversational interviews in each language. All participants were early bilinguals and members of the heterogeneous bilingual speech community in Vancouver, BC, Canada. Chapter \ref{ch2:corpus} documents the motivation, design, and procedures used in the creation of SpiCE. Additionally, a detailed description of the participants is provided. SpiCE is an open-access corpus freely available to anyone interested in the data---researchers, developers, hobbyists, and the general public \citep{johnson_2021_spice}. On its own, SpiCE represents a major contribution to the study of bilingual speech production. 

Chapter \ref{ch3:voice} describes a study on the structure of acoustic voice variation within and across languages for the talkers in the SpiCE corpus. Using a wide array of source and filter acoustic measurements on voiced speech in the conversational interviews, Chapter \ref{ch3:voice} addresses crosslinguistic similarity in three ways. First, the distributions of each measurement were compared across languages on a by-talker basis using Cohen's \textit{d}. The vast majority of comparisons resulted in trivial differences, indicating that talkers were mostly internally consistent. Where consistent differences emerged, they mostly aligned with prior work---Cantonese tended to have lower fundamental frequency and be associated with breathier (or less creaky) voice quality than English \citep{}. 

Second, a series of principal components analyses (PCAs) were run for each talker and language pair. In broad terms, the PCAs bore remarkable similarities in component structure and variance accounted for, regardless of talker and language, given prior work in this domain \citep{}. The PCAs were then subjected to canonical correlation analyses to elucidate how much of the lower dimensional structure in one PCA could be accounted for by the other PCA---that is, how much redundancy there is between two PCAs---and vice versa. The result of this analysis clearly demonstrates that talkers bear the most similarity to themself across languages, compared to any kind of across-talker comparison. While there is some variation in the degree of similarity, the takeaway from this chapter is that voices can largely be thought of as ``auditory faces.''

Chapter \ref{ch4:uniformity} presents a second corpus study, focused on describing and analyzing the structure of phonetic category variation within and across languages for long-lag stops. Leveraging the uniformity framework \citep{chodroff_2017_structure}, Chapter \ref{ch4:uniformity} demonstrates that there is some structure to the relationship between voice-onset time patterns, but that the account is far less compelling when compared to prior work on English \citep{}. Talkers were wildly inconsistent with respect to the expected ordinal relationships between means for the three stop categories within languages. That is, very few talkers produced /p/ with shorter voice onset time than /t/ or /k/, and likewise between /t/ and /k/. The second phase of the analysis considered pairwise correlations of category means for voice onset time within and across languages. Again, the results were far from compelling. While there were consistent moderate correlations for within-language comparisons---especially for English---the across-language correlations were weaker or non-significant. Their presence indicates some degree of structure but does not make for tidy conclusions. 

Chapter \ref{ch4:uniformity} ends with a Bayesian linear mixed-effects model with two primary goals. First, estimate the effect of language while accounting factors known to influence voice onset time. Second, evaluate the sources of variability within the model. The model makes clear that there is a small but consistent effect, such that English stops are longer than their Cantonese counterparts. The model also demonstrates that differences between talkers and words account for far more variation than differences in language and place of articulation effects. Along with the ordinal relationships and correlation analysis, Chapter \ref{ch4:uniformity} depicts both talker and language influences on the structure of voice onset time. 

\section{General discussion}\label{ch5:sec:discussion}

Each of the chapters includes a discussion that deals with topics central to the study at hand. This general discussion focuses on the bigger picture and implications that the two studies have for perception, as foreshadowed in Chapter \ref{ch1:intro}. While there is substantial similarity across languages in both voice variability and the structure of long-lag stops, each study leaves room for both language and talker-indexical influences. 

In Chapter \ref{ch3:voice}, this ``room'' shows up in a couple of ways---first, a sizable subset of talkers in the SpiCE corpus maintains a crosslinguistic difference on measures associated with pitch and breathiness. When a difference exists, Cantonese tends to be lower and breathier. Additionally, while talkers exhibit strong similarity with themselves across languages, the redundancy still varies. That is, some talkers are more similar across languages than others, and none show perfect redundancy. In Chapter \ref{ch4:uniformity}, this ``room'' shows up in the juxtaposition between moderate correlations and a small but consistent difference in voice-onset time across languages. The moderate correlations for homorganic cross-language pairs indicate some level of uniformity in production. Yet, at the same time, the crosslinguistic difference whereby English is characterized by longer VOT shows linguistic influences. 

\hl{work-in-progress!}

% Conclusions, or what Chapters 3 and 4 tell us about:
% how variation is structured across languages
% what languages share
% where languages differ
% what exists in the signal for perception to rely on
% answers to Orena et al.'s speculations

\section{Limitations}\label{ch5:sec:limitations}

% Limitations and caveats
% Based on a heterogeneous group of 34 bilinguals, controlled tightly in some ways but not in others
% Variable behavior not accounted for -- lots of individual differences
% Forced alignment
% Methods more exploratory
% Generates hypotheses and predictions for perception but does not necessarily answer them

\section{Current and future directions}\label{ch5:sec:directions}
% Current perception research
% Promising directions

\section{Conclusion}\label{ch5:sec:conclusion}Take-home points:
% Learning a voice is to learn how it varies.
% The same goes for speech sound categories, suggesting Bullock & Toribio were really onto something.
% Identification likely relies heavily on voice variation structure, but systematic shifts (i.e., in F0 or speech categories) are not off the table.
% just scratching the surface here, but data sets like SpiCE will help push the field forward


\endinput % -------------------------------------------------------- %
